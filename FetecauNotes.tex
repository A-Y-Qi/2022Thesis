\documentclass{article}
\usepackage[utf8]{inputenc}
\usepackage{amsfonts}
\usepackage{array}
\usepackage{graphicx}
\usepackage{hyperref}
\usepackage{amsmath}
\usepackage{amssymb}
\usepackage{xcolor}
\usepackage{commath}
\usepackage{color}
\usepackage{bm}
\usepackage{mathtools}
\usepackage[left=1.00in, right=1.00in, top=1.00in, bottom=1.00in]{geometry}
\usepackage{enumitem}
\renewcommand{\baselinestretch}{1.5}
\newcommand{\I}{\mathbb{I}}
\usepackage{subcaption}

\title{Fetecau Model Part 1}
\author{Andree Qi }
\date{November 2020}

\begin{document}

\maketitle

\section{Introduction}
\textbf{\large {1D model:}}

$$\partial_tu^++\partial_x(\gamma
u^+)=-\lambda^+[u^+,u^-]+\lambda^-[u^+,u^-]u^-$$ 
$$\partial_tu^-+\partial_x(\gamma
u^-)=-\lambda^+[u^+,u^-]\lambda^-[u^+,u^-]u^-$$
\noindent
$\bold{u^+}$: right moving density of individual  \hspace{1cm} $\bold{u_-}$: left moving density of individual\\
$\boldsymbol{\gamma}$: constant speed   \hspace{1cm} ${\boldsymbol{\lambda^+(\lambda_-)}}$:turning rate of original at right(left) to the left(right)

\section{Model Description}
\textbf{\large {2D model:}}
$$\partial_t u 
+\gamma \boldsymbol{e}_\phi \cdot \nabla_x u
=-\lambda(\boldsymbol{x}, \phi)u
+\int_{-\pi}^{\pi}T(\boldsymbol{x},\phi)u(\boldsymbol{x},\phi^{'},t)d\phi^'$$
$u(\boldsymbol{x},\phi^{'},t)$ is the density of individuals at $\boldsymbol{x}=(x,y)$moving in the direction $\phi (-\pi,\pi])$ \\
$\boldsymbol{\gamma}$: constant speed in the direction $\boldsymbol{e}_{/phi}=(cos\phi,sin\phi)$ \hspace{1cm}\\
$\lambda(\boldsymbol{x}, \phi)$: reorientation rate of individuals at ($\boldsymbol{x},\phi$)\\
$T(\boldsymbol{x},\phi)$: the rate of individuals at $x$ that changes their direction from $\phi^'$to $\phi$
$$\lambda(\boldsymbol{x}, \phi)=\int_{-\pi}^{\pi}T(\boldsymbol{x},\phi)u(\boldsymbol{x},\phi^{'},t)d\phi^'$$
\\
\noindent
\textbf{\large {Assumptions on $\lambda(\boldsymbol{x},\phi)$ and $T(\boldsymbol{x},\phi^{'},\phi)$}:}\\
\noindent
\large{1. Distance from neighbour}\\
{\color{blue} \textbf{Distance kernel: what is distance kernel?}}\\
$$K^d_j(\boldsymbol{x})=\frac{1}{A_j}e^{-(\sqrt{x^2+y^2}-d_{j})^2/m^2_j}$$ with $j=r,al,a$ and $d_j$ represents the ranges. \\
$A_j$ ensures that each kernel integrates to 1:
$$A_{j}=\pi m_{j}(m_{j}e^{-d_{j}^{2}/m_{j}^{2}}+\sqrt{\pi}d_{j}+\sqrt{\pi}d_{j}erf(d_{j}/m_{j}))$$
where erf is error rate function. \\
{\color{blue} \textbf{Figure 1b: superposition?}}\\
\noindent
\large {2. Neighbour's orientation}\\
Can be described using rotation kernels dependent on the:\\
\textbf{ reference individual: }movement direction $\phi$ and position $\boldsymbol{x}$\\
\textbf{ reference individual's neighbour: }movement direction $\theta$ and position $\boldsymbol{s}$\\
\\
\noindent
\textbf{\large {Modeling the turning rate $\lambda$}:}\\
$q_j$ with $j=al,a,r$is a constant represents the strength of alignment, attraction and repulsion.
\\
\textbf{Alignment:}\\
\textbf{Less} likely to change direction if their neighbours are going with the \textbf{same direction}.\\
\textbf{More} likely to turn if their neighbours are going with \textbf{opposite direction}\\
$$K^{o}_{al}(\theta;\phi)=\frac{1}{2\pi}(-cos(\phi-\theta)+1)$$
If $\theta$ is very close to $\phi$, then $cos(\phi-\theta)$ will be closer to 1, and $(-cos(\phi-\theta)+1)$ will become 0.  $K^{o}_{al}(\theta;\phi)=0$\\
If $\theta$ is completely opposite to $\phi$, then $cos(\phi-\theta)$ will be closer to -1, and $(-cos(\phi-\theta)+1)$ will become 2. $K^{o}_{al}(\theta;\phi)=1/\pi$\\
o and d represents distance kernel and rotation kernel.
$$\lambda_{al}(\boldsymbol{x},\phi)=
q_{al}\int_{\mathbb{R}^2}\int^\pi_{-\pi}K^d_{al}(\boldsymbol{x}-\boldsymbol{s})K^o_{al}(\theta;\phi)u(\boldsymbol{s},\theta,t)d\theta d\boldsymbol{s}$$
\\
\\
\textbf{Attraction:}\\
Rotation mechanism depend on $\boldsymbol{x}$and $\boldsymbol{s}$ positions. Let $\boldsymbol{s}$ - $\boldsymbol{x}$=($s_x,s_y$), then we can get $\psi$ by 
$$cos(\psi)=\frac{s_x}{\sqrt{s_x^2+s_y^2}},sin(\psi)=\frac{s_y}{\sqrt{s_x^2+s_y^2}}$$
then we get 
$$K^{o}_{a}(\boldsymbol{s};\boldsymbol{x},\phi)=\frac{1}{2\pi}(-cos(\phi-\psi)+1)$$
If $\psi$ is very close to $\phi$, then $cos(\phi-\psi)$ will be closer to 1, and $(-cos(\phi-\psi)+1)$ will become 0.  $K^{o}_{a}(\boldsymbol{s};\boldsymbol{x},\phi)=0$\\
If $\psi$ is completely opposite to $\phi$, then $cos(\phi-\psi)$ will be closer to -1, and $(-cos(\phi-\psi)+1)$ will become 2.
$K^{o}_{a}(\boldsymbol{s};\boldsymbol{x},\phi)=1/\pi$
$$\lambda_{a}(\boldsymbol{x},\phi)=
q_{a}\int_{\mathbb{R}^2}\int^\pi_{-\pi}K^d_{a}(\boldsymbol{x}-\boldsymbol{s})K^{o}_{a}(\boldsymbol{s};\boldsymbol{x},\phi)u(\boldsymbol{s},\theta,t)d\theta d\boldsymbol{s}$$
\\
\\
\textbf{Repulsion:}\\
Totally opposite to the attraction pattern where
$$K^{o}_{r}(\boldsymbol{s};\boldsymbol{x},\phi)=\frac{1}{2\pi}(cos(\phi-\psi)+1)$$
where when the angle is closer, the more likely they are going to turn, an vice versa. 
$$\lambda_{r}(\boldsymbol{x},\phi)=
q_{r}\int_{\mathbb{R}^2}\int^\pi_{-\pi}K^d_{r}(\boldsymbol{x}-\boldsymbol{s})K^{o}_{r}(\boldsymbol{s};\boldsymbol{x},\phi)u(\boldsymbol{s},\theta,t)d\theta d\boldsymbol{s}$$
\\
\textbf{Sum All Together}\\
$$\lambda(\boldsymbol{x},\phi)=\lambda_{al}(\boldsymbol{x},\phi)+\lambda_{a}(\boldsymbol{x},\phi)+\lambda_{r}(\boldsymbol{x},\phi)$$

\noindent
\textbf{\large {Modeling the reorientation terms}}
\\
$T(\boldsymbol{x},\phi^{'},\phi)$:\\
Turning probability function:
$\omega(\phi^{'}-\phi,\phi^{'}-\theta)$.\\
$\omega$ represents the possibility of turning rate from direction $\phi^{'}$ to $\phi$ as a result of interactions with individuals located(moving) at direction $\theta$.
And we get reorientation terms similar to the turning rate.\\
$$T_{al}(\boldsymbol{x},\phi^{'},\phi)
=q_{al}\int_{\mathbb{R}^2}\int^{\pi}_{\pi}K^{d}_{al}(\boldsymbol{x}-\boldsymbol{s})K^{o}_{al}(\theta;\phi)\omega_{al}(\phi^{'}-\phi,\phi^{'}-\theta)
u(\boldsymbol{s},\theta,t)d\theta d\boldsymbol{s}$$
$$T_{a}(\boldsymbol{x},\phi^{'},\phi)
=q_{a}\int_{\mathbb{R}^2}\int^{\pi}_{\pi}K^{d}_{a}(\boldsymbol{x}-\boldsymbol{s})K^{o}_{a}(\boldsymbol{s};\boldsymbol{x},\phi^{'})\omega_{a}(\phi^{'}-\phi,\phi^{'}-\theta)
u(\boldsymbol{s},\theta,t)d\theta d\boldsymbol{s}$$
$$T_{r}(\boldsymbol{x},\phi^{'},\phi)
=q_{r}\int_{\mathbb{R}^2}\int^{\pi}_{\pi}K^{d}_{r}(\boldsymbol{x}-\boldsymbol{s})K^{o}_{r}(\boldsymbol{s};\boldsymbol{x},\phi^{'})\omega_{r}(\phi^{'}-\phi,\phi^{'}-\theta)
u(\boldsymbol{s},\theta,t)d\theta d\boldsymbol{s}$$
And similar to the turning rate $\lambda$, the total term of $T$ is:
$$T(\boldsymbol{x},\phi^{'},\phi)
=T_{al}(\boldsymbol{x},\phi^{'},\phi)+T_{a}(\boldsymbol{x},\phi^{'},\phi)+T_{r}(\boldsymbol{x},\phi^{'},\phi)$$
{\color{blue} \textbf{What is the reorientation term? What does this term represent?}}\\
\textbf{\large {Modeling the probability function $\omega_{al}, \omega_{a}$ and $\omega_{r}$}}
$$\omega(\phi^{'}-\phi,\phi^{'}-\theta)
=g_{\sigma}(\phi^{'}-\phi-v(\phi^{'}-\theta))$$
$g_\sigma$ is an approximation of the delta function with width $\sigma$\\
$v$ is the turning function with highest possibility of an individual moving in the direction $\phi^{'}$ turns to $\phi=\phi{'}-v(\phi^{'}-\theta)$
{\color{blue} \textbf{What is turning function?}}\\
$\sigma>0$ is the uncertainty of turning. \\
Possible expressions for $g_\sigma$ are a \textbf{periodic Gaussian function} or a \textbf{Normalized step function}:
$$g_{\sigma}(\theta)=
\frac{1}{\sqrt{\pi}\sigma}\sum_{z\in \mathbb{Z}}e^{{-(\frac{\theta+2\pi z}{\sigma})}^2}, \theta \in (-\pi,\pi)$$
what is z above? 
\begin{equation}
  g_{\sigma} =
    \begin{cases}
    \frac{1}{2\sigma}, |\theta|<\sigma\\
      0 \sigma<|\theta|<=\pi
    \end{cases}       
\end{equation}
Turning function $v$:
$$v(\theta)=\kappa sin(\theta)$$
$$v(\theta)=\kappa \theta), -1<=\kappa<=1$$
Sign of $\kappa$ is very important. if $\kappa$ is negative, then it is \textbf{repulsive turning}, if $\kappa$ is positive, then it is \textbf{attracting-like turning}. 

\section{Numerical results}
Use Fourier Methods to calculate the convolution integral for space and angle. \\
{\color{blue} \textbf{What does it mean by "performs a multiplication in the discrete Fourier transform"?} $$\widehat{K*u(l)}=\hat{K}(l)\hat{u}(l)$$ }\\
Apply 2D Fourier transform to $\gamma \boldsymbol{e}_\phi \cdot \nabla_x u$ and then obtain $$\gamma(cos \phi l_1 + sin \phi l_2)\hat{u}$$ 
where $l_1 and l_2$ are horizontal and vertical components of the wave number.
\par
Use the \textbf{4th Runge-Kutta Method} to solve for the numerical solution.\\
\textbf{Space discretization:} rectangular grid on $[-L/2,L/2)\times [-L/2,L/2)$ with $N^2$ points, with $\Delta x=\Delta y=L/N$\\
\textbf{Angle discretization:} equidistant grid on [$-\pi$, $\pi$) with M points, $\Delta \phi = 2 *\pi / M$\\

To avoid aliasing, all multiplications of Fourier modes are done on an extended spatial grid of size $(\frac{3}{2} N)^2$ and an angular gird of size $\frac{3}{2} M$. 

\textbf{Discrete Convolutions to compute turning rates}\\
The integral defining $\lambda _ {al}$ is trivial to compute its discrete Fourier Spectrum. \\
The integrals that define $\lambda _ a$ and $\lambda _ r$ are similar and can represent a convolution in space only. What we get is 
$$\lambda_{r}(\boldsymbol{x},\phi)=
q_{r}\int_{\mathbb{R}^2}K^d_{r}(\boldsymbol{x}-\boldsymbol{s})K^{o}_{r}(\boldsymbol{s};\boldsymbol{x},\phi)\int^\pi_{-\pi} u(\boldsymbol{s},\theta,t)d\theta d\boldsymbol{s}$$
The $\theta$ integral is the zero mode of u, and the remaining space integral is a convolution. 
\textbf{Calculation of the reorientation terms included in T}\\
We can change $T_a(\boldsymbol{x}, \phi ', \phi)$ and $T_r(\boldsymbol{x}, \phi ', \phi)$ in space only. For every $\phi$ and $\phi '$ fixed, then we get 
$$\omega_{a}(\phi-\phi',\phi ' -\psi)=g \sigma_{a}(\phi ' - \phi - \kappa_{a}sin(\phi ' -\psi))$$
$$=g \sigma_{a}(\phi ' - \phi - \kappa_{a}(sin\phi 'cos\psi-cos\phi ' sin\psi)$$
$$=g \sigma_{a}(\phi ' - \phi - \kappa_{a}(sin\phi '\frac{s_{x}}{s_{x}^{2}+s_{y}^{2}}-cos\phi ' \frac{s_{y}}{s_{y}^{2}+s_{x}^{2}}))$$
which is a function of $\boldsymbol{x}-\boldsymbol{s}$, which means it is a convolution is the space only. 


\end{document}
