\documentclass[12pt]{article}
\usepackage[utf8]{inputenc}
\usepackage{indentfirst}
\usepackage{amsfonts}
\usepackage{array}
\usepackage{graphicx}
\usepackage{hyperref}
\usepackage{amsmath}
\usepackage{amssymb}
\usepackage[dvipsnames]{xcolor}
\usepackage{commath}
\usepackage{color}
\usepackage{bm}
\usepackage{mathtools}
\usepackage[left=1.00in, right=1.00in, top=1.00in, bottom=1.00in]{geometry}
\usepackage{enumitem}
\setlength\parindent{24pt}
\renewcommand{\baselinestretch}{1}
\newcommand{\I}{\mathbb{I}}
\usepackage{subcaption}
\usepackage{float}

\numberwithin{equation}{section}


%=========Defined Equations===========%
\def\transeq{u_{t}+\gamma \mathbf{e}_{\phi} \cdot \nabla_{\boldsymbol{x}} u=0}

\def\fourieru{\hat{u}_{k_{1}, k_{2}}(\phi, t)=\alpha \int_{x_{2}=-\pi}^{\pi} \int_{x_{1}=-\pi}^{\pi}u\left(x_{1}, x_{2}, \phi, t\right) \exp \left(- i\left[k_{1} x_{1}+k_{2} x_{2}\right] \right) \mathrm{d} x_{1} \mathrm{~d} x_{2}}

\def\invfourieru{u\left(x_{1}, x_{2}, \phi, t\right)=\beta \sum_{k_{1}=\frac{N}{2}}^{\frac{N}{2}}\sum_{k_{2}=\frac{N}{2}}^{\frac{N}{2}} \hat{u}_{k_{1}, k_{2}}(\phi, t) \exp \left( i\left[k_{1} x_{1}+k_{2} x_{2}\right] \right)}

\def\F_transeq{\frac{\partial \hat{u}_{k_{1}, k_{2}}(\phi, t)}{\partial t}+ i \gamma\left[k_{1} \cos (\phi)+k_{2} \sin (\phi)\right] \hat{u}_{k_{1}, k_{2}}(\phi, t)=0}


\title{Linear Part Draft}
\author{Yuelin Qi}
\date{May 2022}

\begin{document}
\maketitle
\section{Exponential Time Differencing Scheme}
In this chapter, the main time stepping method is introduced and applied to solve the equation over time. The method is applied to solve an ordinary differential equation (ODE) involving both linear and non-linear terms, which can be generalized as:
\begin{equation}\label{eq:EDT general}
\frac{du(t)}{dt}=Lu(t)+N(u(t),t)
\end{equation}
where $L$ is a linear operator acting on $u(t)$ and $N(u(t),t)$ is the non-linear term. 
To solve the ODE in \eqref{eq:EDT general}, we first state an integrating factor $\mu=\exp(-Lt)$ and multiply both sides of \eqref{eq:EDT general} by $\mu$ to get a new equation
\begin{equation}\label{eq:after_integrating_factor}
\frac{d}{dt}(\exp(-Lt) u(t))=\exp(-Lt)N(u(t),t)
\end{equation}
Suppose at the current time step $t\prime$, the non-linear term $N(u(t\prime),t\prime)$ is known as a constant and if we integrate both sides over $(t\prime,t\prime+\Delta t)$, the equation \eqref{eq:after_integrating_factor} becomes:
\begin{equation}
\exp(-Lt) u(t)\Big | ^{t\prime+\Delta t}_{t\prime}=N(u(t\prime),t\prime) \int^{t\prime+\Delta t}_{t\prime}\exp(-Lt) dt 
\end{equation}
Finally, the density at the new time step $u(t\prime+\Delta t)$ can be calculated as
\begin{equation}
u(t\prime+\Delta t)= \exp (L \Delta t)u(t\prime)+(\exp(L\Delta t)-1)N(u(t\prime),t\prime)
\end{equation}


\subsection{Calculating the Transportation Term}
This section will mainly be focusing on simulating the transportation term using Exponential Time Differencing (ETD) scheme. For the interaction terms all equal to zero, the equation can be simplified as
\begin{equation}\label{eq:transeq}
\transeq
\end{equation}
To simulate the gradient of $u$ over the space, we can apply semidescrete Fourier transform with an assumption that the function $u$ is bounded over $[-\pi,\pi]$, then 
\begin{equation}\label{eq:semi_fourier}
\fourieru
\end{equation}
where $k_1, k_2$ are the wave number corresponding to $x_1,x_2$. The inverse Fourier transform is
\begin{equation}\label{eq:semi_invfourier}
\invfourieru
\end{equation}
where $N$ is the discrete grid number and $\alpha \beta = \frac{1}{4\pi^2}$. If Fourier transform over the space is applied to \eqref{eq:transeq}, then $\nabla_{x} \hat{u}=(ik_1\hat{u}, ik_2\hat{u})$. Therefore, the equation becomes
\begin{equation}\label{eq:F_transeq}
\F_transeq
\end{equation}
 Suppose that $k_1, k_2$ be known constants, and discretize $\phi$ over a grid between $[-\pi,\pi]$ with an even number $m$ points such that $\phi_j=\frac{2 \pi}{m}j$ with $j=-\frac{m}{2} \cdots \frac{m}{2}-1$, the linear operator at the angle $\phi_j$ would be $L=-i\gamma [k_1cos(\phi_j)+k_2sin(\phi_j)]$. If we define $\hat{u}_{k_{1}, k_{2}, j} \equiv \hat{u}_{k_{1}, k_{2}}\left(\phi_{j}, t\right)$, \eqref{eq:F_transeq} can be written as and ODE
 \begin{equation}\label{eq:M_transeq}
\frac{\mathrm{d} \hat{u}_{k_{1}, k_{2}, j_{1}}(t)}{\mathrm{d} t}=\sum_{j_{2}=-\frac{m}{2}}^{\frac{m}{2}-1} M_{p_{1}, p_{2}} \hat{u}_{k_{1}, k_{2}, j_{2}}
\end{equation}
where $j_1,j_2\in (-\frac{m}{2}, \frac{m}{2})$ and $p_1= j_1+\frac{m}{2}$, $p_2= j_2+\frac{m}{2}$ . Then $M$ is a diagonal matrix having entries $M_{p_1,p_2}=L(k_1,k_2,\phi_{j_1})\delta_{j_1,j_2}$  with $\delta_{j_1,j_2}$ be the Dirac delta function. Then if we omit the known constants $k_1, k_2$, the system would become
\begin{equation}\label{eq:matrix_transeq}
\frac{d}{dt}\left[\begin{array}{c}
\hat{u}_{-\frac{m}{2}} \\
\hat{u}_{-\frac{m}{2}+1} \\
\vdots \\
\hat{u}_{\frac{m}{2}-1}
\end{array}\right]
=\left[\begin{array}{cccc}
L(\phi_{-\frac{m}{2}}) & 0& \cdots & 0\\
0 & L(\phi_{-\frac{m}{2}+1}) & \ddots&\vdots  \\
\vdots & \ddots & \ddots & 0\\
0  & \cdots & 0& L(\phi_{\frac{m}{2}-1})
\end{array}\right]\left[
\begin{array}{c}
\hat{u}_{-\frac{m}{2}} \\
\hat{u}_{-\frac{m}{2}+1} \\
\vdots \\
\hat{u}_{\frac{m}{2}-1}
\end{array}\right]
\end{equation}
where 
\begin{equation}
M=\left[\begin{array}{cccc}
L\left(\phi_{-\frac{m}{2}}\right) & 0 & \cdots & 0 \\
0 & L\left(\phi_{-\frac{m}{2}+1}\right) & \ddots & \vdots \\
\vdots & \ddots & \ddots & 0 \\
0 & \cdots & 0 & L\left(\phi_{\frac{m}{2}-1}\right)
\end{array}\right]
\end{equation}
To solve the system \eqref{eq:matrix_transeq}, we can apply the ETD scheme, where we set the integrating factor $\mu=\exp(-Mt)$, the system cam be simplified as 
\begin{equation}\label{eq:int_transeq}
\frac{\mathrm{d}}{\mathrm{d} t}(\exp (-M t) \hat{u})=0
\end{equation}
If we integrate both sides over the current time step $t\prime$ and the new time step $t\prime+\Delta t$ to get 
\begin{equation}\label{eq:int_transeq_2}
\exp(-M(t\prime+\Delta t))\hat{u}(t\prime+\Delta t)-\exp(-M(t\prime))\hat{u}(t\prime)=0
\end{equation}
the right hand side remains zero since the non-linear term equals to zero. Then to solve for $u(t\prime+\Delta t)$ the equation becomes
\begin{equation}\label{eq:lin_fourier_u_new}
\hat{u}(t\prime+\Delta t)=\exp(M\Delta t)\hat{u}(t\prime)
\end{equation}
Finally, the inverse Fourier transform (stated in \eqref{eq:semi_invfourier} ) is applied to \eqref{eq:lin_fourier_u_new} to find the density function $u(t\prime+\Delta t)$ in real space. 
\subsection{Simulation of the Transportation Term}
In this section, we are going to apply the EDT scheme to an initial condition of $u$
\begin{equation}
u_0(x_1,x_2,\phi)=\exp(\cos(x_1)+\cos(x_2))
\end{equation}
which is shown in Figure \ref{fig:Initial_condition}. 
\begin{figure}[H]
    \centering
    \includegraphics[width=\textwidth]{Initial_Condition.png}
    \caption[Network2]%
    {{\small The Initial Condition $u_0(x_1,x_2,\phi)=\exp(\cos(x_1)+\cos(x_2)) $}}
    \label{fig:Initial_condition}
\end{figure}
The initial condition $u_0$ stays the same for different $\phi$. But as the time variable $t$ gets larger, since the population is moving toward different directions, the density plot is also different at different $\phi$ (Figure \ref{fig:trans_term_diff_phi}).
\begin{figure}[H]
        \centering
        \begin{subfigure}[b]{\textwidth}
            \centering
            \includegraphics[width=0.49\textwidth]{N32t50phi8.png}
        	\hfill
            \includegraphics[width=0.49\textwidth]{N32t100phi8.png}
            \caption[]%
            {{\small The Population Density Plot of $u(x_1, x_2, \phi=-\frac{\pi}{2})$ at $t=0.5,1$}}    
        \end{subfigure}
        \vskip\baselineskip
        \begin{subfigure}[b]{\textwidth}   
            \centering 
            \includegraphics[width=0.49\textwidth]{N32t50phi16.png}
            \hfill
            \includegraphics[width=0.49\textwidth]{N32t100phi16.png}
            \caption[]
            {{\small The Population Density Plot of $u(x_1, x_2, \phi=0)$ at $t=0.5,1$}}    
        \end{subfigure}
        \vskip\baselineskip
        \begin{subfigure}[b]{\textwidth}   
            \centering 
            \includegraphics[width=0.49\textwidth]{N32t50phi20.png}
            \hfill
            \includegraphics[width=0.49\textwidth]{N32t100phi20.png}
            \caption[]
            {{\small The Population Density Plot of $u(x_1, x_2, \phi=\frac{\pi}{4})$ at $t=0.5,1$}}    
        \end{subfigure}
        \caption{\small The Population Density Plot of $u(x_1, x_2, \phi)$ with different $\phi$ at different $t$}
        \label{fig:trans_term_diff_phi}
\end{figure}

We can also look at the whole population density plot over different angles, where define the whole population $U(x_1, x_2,t)$ as
\begin{equation}
U(x_1, x_2,t)=\int^{\pi}_{-\pi}u(x_1, x_2, \phi,t)d\phi
\end{equation}
Also, let the arrows at the position $x_1, x_2$ represent the angle $\phi_{max}$ with the largest population. The thickness of the arrows varies with different density, where the thicker arrows represent a greater magnitude of density at $u(x_1, x_2,\phi_{max},t)$. The plot is shown in Figure \ref{fig:U_tran_term_only}. In Figure \ref{fig:U_tran_term_only}, the initial population is concentrated in the center. Then the population starts to spread out from the center toward different directions. 
\begin{figure}[H]
        \centering
        \begin{subfigure}[b]{0.49\textwidth}
            \centering
            \includegraphics[width=\textwidth]{Contour_U_t0.png}
            \caption[Network2]%
            {\small when $t=0$}    
        \end{subfigure}
        \hfill
        \begin{subfigure}[b]{0.49\textwidth}  
            \centering 
            \includegraphics[width=\textwidth]{Contour_U_t20.png}
            \caption[]%
            {\small when $t=0.2$}    
        \end{subfigure}
        \vskip\baselineskip
        \begin{subfigure}[b]{0.49\textwidth}   
            \centering 
            \includegraphics[width=\textwidth]{Contour_U_t40.png}
            \caption[]%
            {\small when $t=0.4$}    
        \end{subfigure}
        \hfill
        \begin{subfigure}[b]{0.49\textwidth}   
            \centering 
            \includegraphics[width=\textwidth]{Contour_U_t60.png}
            \caption[]
            {\small when $t=0.6$} 
        \end{subfigure}
        \vskip\baselineskip
        \begin{subfigure}[b]{0.49\textwidth}   
            \centering 
            \includegraphics[width=\textwidth]{Contour_U_t80.png}
            \caption[] {\small when $t=0.8$}    
        \end{subfigure}
        \hfill
        \begin{subfigure}[b]{0.49\textwidth}   
            \centering 
            \includegraphics[width=\textwidth]{Contour_U_t100.png}
            \caption[]%
            {\small when $t=1.0$}    
        \end{subfigure}
        \caption[ The average and standard deviation of critical parameters ]
        {\small the Attraction Kernel} 
        \label{fig:U_tran_term_only}
\end{figure}



\end{document}
